\documentclass[12pt]{article}
\usepackage[left=3cm,right=2cm,top=3cm,bottom=2cm,a4paper,includehead,includefoot]{geometry} 
\usepackage{natbib}
\usepackage{amsmath}

%\VignetteIndexEntry{Vignette on miRtest package}

% TITLE PAGE
\title{miRtest v. 1.0 Package Vignette}
\author{Stephan Artmann <stephan-artmann@gmx.net>,\\Klaus Jung,\\Tim Bei\ss barth}
\date{G\"ottingen 2011}


% BEGIN OF DOCUMENT
\usepackage{Sweave}
\begin{document}
\maketitle
\tableofcontents
\section{Introduction}
High-throughput measurements of gene expression are gaining popularity. So are microRNA analyses. The `miRtest' package \cite[] {detection_artmann_submitted} intends to help researches find regulated expressed miRNAs between two groups.

`miRtest' tries to improve power when testing for differentially regulated miRNAs by incorporation of their regulated gene sets' expression data.

miRNA-wise testing is done with the linear models implemented in the `limma' package \cite[] {smyth_gordon_k_linear_2004}. For gene set testing, different procedures can be chosen from: the self-contained tests `globaltest' \cite[] {goeman_global_2004}, `GlobalAncova' \cite[]{mansmann_testing_2005,hummel_globalancova:_2008}, `RepeatedHighDim' \citep{jung_comparison_submitted,brunner_repeated_2009}, the competitive tests `ROAST' \cite[]{wu_roast:_2010} and `Romer' \cite[]{majewski_opposing_2010} as well as non-rotation enrichment tests.
\section{Simple Example}
The main function of `miRtest' is `miR.test'. It requires the user to supply an expression matrix $\boldsymbol{X}$ of miRNAs with miRNAs in its rows and microarray samples in its columns. Additionally, the procedures require an analogous matrix $\boldsymbol{Y}$ of mRNA expression values. Finally, a data.frame $\boldsymbol{A}$ is necessary: it defines which mRNAs are attacked by which miRNA.
To begin with, we will generate random expression data: miRNA expression matrix $\boldsymbol{X}$, mRNA expression matrix $\boldsymbol{Y}$ and an allocation data.frame $\boldsymbol{A}$.
\begin{Schunk}
\begin{Sinput}
> set.seed(1)
> X = rnorm(24)
> dim(X) = c(3, 8)
> rownames(X) = 1:3
\end{Sinput}
\end{Schunk}
In this synthetic experiment, 8 microarray replicates are present with three miRNAs on each. Additionally, we need a corresponding matrix $\boldsymbol{Y}$ for mRNAs. Here we assume we have 20 mRNAs and 10 microarray replicates:
\begin{Schunk}
\begin{Sinput}
> Y = rnorm(200)
> dim(Y) = c(20, 10)
> rownames(Y) = 1:20
\end{Sinput}
\end{Schunk}
Now we need to define what we want to test for. We shall concentrate on two-group testing, i. e. the search for miRNAs differentially expressed between two groups $1$ and $2$. For other designs see Section~\ref{designs}. Let's say both groups are of equal sample size in miRNA and mRNA microarrays:

\begin{Schunk}
\begin{Sinput}
> group.miRNA = factor(c(1, 1, 1, 1, 2, 2, 2, 2))
> group.mRNA = factor(c(1, 1, 1, 1, 1, 2, 2, 2, 2, 2))
\end{Sinput}
\end{Schunk}

Next, we need the allocation information. In most databases it is provided as a data.frame $\boldsymbol{A}$, where the first column contains mRNAs and the second miRNAs. Each row of $\boldsymbol{A}$ indicates which mRNA is targeted by which miRNA. Let's say that miRNA $1$ has nine target genes and miRNA $2$ the remaining ones. The gene set of miRNA $3$ will be empty.
\begin{Schunk}
\begin{Sinput}
> library(miRtest)
> miR = c(rep(1, 9), c(rep(2, 8)))
> mRNAs = 1:17
> A = data.frame(mRNAs, miR)
> A
\end{Sinput}
\begin{Soutput}
   mRNAs miR
1      1   1
2      2   1
3      3   1
4      4   1
5      5   1
6      6   1
7      7   1
8      8   1
9      9   1
10    10   2
11    11   2
12    12   2
13    13   2
14    14   2
15    15   2
16    16   2
17    17   2
\end{Soutput}
\end{Schunk}

Finally, the function `miR.test' is called which does the testing.
\begin{Schunk}
\begin{Sinput}
> set.seed(1)
> P = miR.test(X, Y, A, group.miRNA, group.mRNA)
> P
\end{Sinput}
\begin{Soutput}
    miRtest
1 0.7208223
2 0.3157580
3        NA
\end{Soutput}
\end{Schunk}
Note that for the empty gene set `NA' was returned.
\pagebreak
\section{Choice of Gene Set Tests}
The `gene.set.test' argument in miR.test takes a vector of strings. These are the gene set tests that shall be applied. The default is the `romer' test as it is competitive and compensates for inter-gene correlations. The different gene set tests available are:\\~\\
\begin{tabular}{lr}
\hline
\textbf{Test}&\textbf{Name in miR.test}\\
\hline
\\
\textbf{Self-contained}&\\
`globaltest' \cite[]{goeman_global_2004}&"globaltest"\\
`GlobalAncova'&"GA"\\
\citep{mansmann_testing_2005,hummel_globalancova:_2008}&\\
`RepeatedHighDim' \citep{brunner_repeated_2009}&"RHD"\\
\\
\textbf{Competitive}\\
Kolm. Smirnov test on gene ranks&"KS"\\
Wilcoxon test on gene ranks&"W"\\
Fisher's exact test on gene ranks with 5 \% FDR threshold&"Fisher"\\
`ROAST' \cite[]{wu_roast:_2010}&"roast"\\
`romer' \cite[]{majewski_opposing_2010}&"romer"\\
\hline
\end{tabular}
\subsection{Faster Algorithm}
The specification of other gene set tests in miR.test is therefore rather simple. To obtain faster results than with the default `romer' rotation test, the Wilcoxon two-sample test based on gene ranks is recommended:
\begin{Schunk}
\begin{Sinput}
> P.gsWilcox = miR.test(X, Y, A, group.miRNA, group.mRNA, gene.set.tests = "W")
> P.gsWilcox
\end{Sinput}
\begin{Soutput}
    miRtest
1 0.8520745
2 0.3297708
3        NA
\end{Soutput}
\end{Schunk}
\pagebreak
\section{Other Input Formats of Allocation Data}
To make `miR.test' run faster one can specify an allocation matrix instead of the allocation data.frame. Its columns stand for the miRNAs and its rows for the mRNAs. If a mRNA is a target of a miRNA, the corresponding entry is $1$, else it is $0$. An easy way to generate allocation matrices is the `generate.A' function:
\begin{Schunk}
\begin{Sinput}
> A = generate.A(A, X = X, Y = Y, verbose = FALSE)
> A
\end{Sinput}
\begin{Soutput}
   1 2 3
1  1 0 0
2  1 0 0
3  1 0 0
4  1 0 0
5  1 0 0
6  1 0 0
7  1 0 0
8  1 0 0
9  1 0 0
10 0 1 0
11 0 1 0
12 0 1 0
13 0 1 0
14 0 1 0
15 0 1 0
16 0 1 0
17 0 1 0
18 0 0 0
19 0 0 0
20 0 0 0
\end{Soutput}
\end{Schunk}
To use the allocation matrix, we need to set `allocation.matrix=TRUE' in `miR.test':
\begin{Schunk}
\begin{Sinput}
> set.seed(1)
> P = miR.test(X, Y, A, group.miRNA, group.mRNA, allocation.matrix = TRUE)
> P
\end{Sinput}
\begin{Soutput}
    miRtest
1 0.7208223
2 0.3157580
3        NA
\end{Soutput}
\end{Schunk}

\pagebreak
\section{Other Designs than Two-Group Design \label{designs}}
Primarily, `miRtest' has been designed for two-group comparisons. However, `miRtest' accepts design matrices as used in `limma' \cite[] {smyth_gordon_k_linear_2004}. The only limitation is that `miRtest' takes the second column from `limma's `eBayes' function to calculate final $p$-values. This already allows designs including
\begin{itemize}
 \item covariables and
 \item continuous group/response vectors.
\end{itemize}
Other designs will be implemented in future versions. Regard the following example which shows how to use `miRtest' on such designs. First we create the design matrices
\begin{Schunk}
\begin{Sinput}
> group.miRNA = 1:8
> group.mRNA = 1:10
> covariable.miRNA = factor(c(1, 2, 3, 4, 1, 2, 3, 4))
> covariable.mRNA = factor(c(1, 2, 3, 4, 5, 1, 2, 3, 4, 5))
> library(limma)
> design.miRNA = model.matrix(~group.miRNA + covariable.miRNA)
> design.mRNA = model.matrix(~group.mRNA + covariable.mRNA)
\end{Sinput}
\end{Schunk}
which then we use in `miR.test'
\begin{Schunk}
\begin{Sinput}
> P = miR.test(X, Y, A, design.miRNA = design.miRNA, design.mRNA = design.mRNA, 
+     allocation.matrix = TRUE)
> P
\end{Sinput}
\begin{Soutput}
    miRtest
1 0.5797948
2 0.4981197
3        NA
\end{Soutput}
\end{Schunk}
Note that so far this works only with the competitive gene set tests and `ROAST'.


%%bibliography
\bibliographystyle{natbib}
\bibliography{bibliography.bib}
\end{document}
